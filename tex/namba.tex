%
%   N A M B A   N A M B A   N A M B A
%    N A M B A   N A M B A   N A M B A
%     N A M B A   N A M B A   N A M B A
%

\documentclass[10pt,b5paper,papersize,dvipdfmx]{jsbook}

\usepackage{vuccaken}
\usepackage{vuccaken2019}

% スタイルファイルの読み込みや自作マクロは、
% 最終的には vuccaken2019.sty の中に書いてください。
% とりあえずはここに書いてもらって構いません。


\begin{document} % 以下本文

% - - - - - - - - - - - - - - - - - - - - - - - - %
\kaishititle%
  {流体中の物体に働く揚力}% title
  {物理科学科1回生}% 所属
  {難波潤}% name
% - - - - - - - - - - - - - - - - - - - - - - - - %

% \setcounter{tocdepth}{2} % 目次にどこまで表示するか
% \tableofcontents % 目次出力
% \clearpage % 改ページ

%
\section*{はじめに}
流体は我々の身の周りに存在していながら、我々の目にはみえない不思議な運動をする。ここではこの不思議な運動は一体どのような法則に従っているのかを解説したいと思う。
%
\section{そもそも流体とは?}
流体は、簡単に言えば自由に変形できる物質である。液体と気体がそれにあたり、特定の形を持たず容易に変形し、接する固体の形状によって運動が支配される。流体の特徴として粘性と呼ばれるものがある。これは流体中を運動する物体に対して、流体が変形を妨げようとして抵抗を示す性質である。

%% 参考文献
\begin{sanko}
  \begin{enumerate}
    \item 著者, 本やページの名前, (URL), 出版社, 出版年.
    \item (複数ある場合は追加)
    \item @vuccaken, 物科研HP, \url{rp2017xy.starfree.jp}, 2019.
  \end{enumerate}
\end{sanko}


\end{document}
%
% ファイトだよ!
%
