%
%    comment for 'fukuda.tex'
%


\documentclass[10pt,b5paper,papersize,dvipdfmx]{jsbook}

\usepackage{vuccaken}
\usepackage{vuccaken2019}

\begin{document}

% - - - - - - - - - - - - - - - - - - - - - - - - %
\kaishititle%
  {fukuda.tex のかうせい}% title
  {校正科学科}% 所属
  {nkym}% name
% - - - - - - - - - - - - - - - - - - - - - - - - %

\section*{はじめに}
早速校正です。反論は歓迎します。2019-10-22

\section{校正}

\begin{itemize}
  \item 真空の誘電率って$\epsilon_o$じゃなくて$\epsilon_0$では?
  \item ナブラは \code|\bigtriangledown| じゃなくて \code|\nabla| で出せます:
    $\bigtriangledown \phi \to \nabla \phi$.
  \item 式番号を参照するとき、かっこ \texttt{()}をつけた方がいいです。
    \ex
    \begin{align}
      F = ma \label{eq:nu-ton}
    \end{align}
    (\ref{eq:nu-ton})式より、、、\par
    または、\texttt{vuccaken2019.sty}でnkymが定義している \code{\siki} コマンド;
    %
    \begin{metatex}[gobble=6]
      \newcommand\siki[1]{(\ref{eq:#1})}
    \end{metatex}
    %
    を使うと楽です。
    \ex \siki{nu-ton}より、、、
  \item 52行目あたりに
    %
    \begin{metatex}[gobble=6]
      $\mathbf{E}:電場、\mathbf{B}:磁場、\mathbf{j}:流束、\mathbf{n}:法線ベクトル、C:任意の閉曲線、S:任意の閉曲面、V:任意の閉曲面Sに囲まれた体積、\rho :電荷密度、c:光速$
    \end{metatex}
    とありますが、面倒臭がらずにちゃんと
    \begin{metatex}[gobble=6]
      $\mathbf{E}$:電場、$\mathbf{B}$:磁場、$\mathbf{j}$:流束、$\mathbf{n}$:法線ベクトル、$C$:任意の閉曲線、$S$:任意の閉曲面、$V$:任意の閉曲面$S$に囲まれた体積、$\rho$:電荷密度、$c$:光速
    \end{metatex}
    と書きましょう。
  \item \code{\footnote}は句点(ピリオド)の前に付けるのが普通っぽいです\footnote{こんな感じで}。
  \item あとは文章中の数式に\code{$ $}を付けるのを忘れないようにするのと、半角全角を間違えないように......
\end{itemize}

\section{好み}
\begin{itemize}
  \item 被積分関数と微小量の間にスペースを入れた方が美しいです。
    \begin{align}
      \int f(x) dx \to \int f(x) \, dx
    \end{align}
  \item
    もっと言えば、微小量を表すdは変数ではなく$\sin$とか$\log$とかの記号と同じようなものなので、イタリック体ではなくローマン体にするのが正しいとされます。
    \begin{align}
      dx \to \mathrm{d}x
    \end{align}
    でもさすがにこれは面倒なので無視するか、\texttt{physics}パッケージを使うのがよいです。以下\texttt{physics}パッケージの例(ソースも見てね):
    \begin{align}
      \int f(x) \dd{x}, \quad 
      \dv{y}{x}, \quad
      \pdv[2]{f}{x}{y}, \quad \text{など}
    \end{align}
  \item \texttt{fukuda.tex}の(1.20)式のようにdisplay styleでカンマを使う時は\code{\quad}でスペースを入れると絶妙です(個人的には):
  \begin{align}
    F_x = - \pdv{U}{x},\quad
    F_y = - \pdv{U}{y},\quad
    F_z = - \pdv{U}{z}
  \end{align}
  \item (2.20)式は、かっこを大きくした方が美しいかと(個人的には):
  \begin{align}
    \mathbf{F}= \left(
      -\pdv{x}U, -\pdv{y}U, -\pdv{z}U
    \right)
  \end{align}
  % OR, use \qty command defined in physics.sty:
  % \begin{align}
  %   \mathbf{F}= \qty(
  %     -\pdv{x}U, -\pdv{y}U, -\pdv{z}U
  %   ), \\
  %   \mathbf{F}= \pqty{
  %     -\pdv{x}U, -\pdv{y}U, -\pdv{z}U
  %   }
  % \end{align}
\end{itemize}


\end{document}


